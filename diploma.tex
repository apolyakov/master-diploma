% Тут используется класс, установленный на сервере Papeeria. На случай, если
% текст понадобится редактировать где-то в другом месте, рядом лежит файл matmex-diploma-custom.cls
% который в момент своего создания был идентичен классу, установленному на сервере.
% Для того, чтобы им воспользоваться, замените matmex-diploma на matmex-diploma-custom
% Если вы работаете исключительно в Papeeria то мы настоятельно рекомендуем пользоваться
% классом matmex-diploma, поскольку он будет автоматически обновляться по мере внесения корректив
%

% По умолчанию используется шрифт 14 размера. Если нужен 12-й шрифт, уберите опцию [14pt]
%\documentclass[14pt]{matmex-diploma}
\documentclass[14pt]{matmex-diploma-custom}
\usepackage{graphicx}
\usepackage{float}
\usepackage{cite}
\usepackage{amsmath,amssymb,amsfonts}
\usepackage{caption}
\usepackage{textcomp}
\usepackage{xcolor}
\usepackage{float}
\usepackage{amssymb}
\usepackage[mathscr]{euscript}

\usepackage{indentfirst}



\newcommand\name[1]{\textsc{#1}}

\begin{document}
% Год, город, название университета и факультета предопределены,
% но можно и поменять.
% Если англоязычная титульная страница не нужна, то ее можно просто удалить.
\filltitle{ru}{
    chair              = {Программная инженерия},
    title              = {Декларативный предметно-ориентированный язык разработки интерфейсов мобильных приложений},
    type               = {coursework},
    position           = {студента},
    group              = 671,
    author             = {Поляков Александр Романович},
    supervisorPosition = {старший преподаватель},
    supervisor         = {Дмитрий Луцив},
    reviewerPosition   = {исследователь в ООО Хуавей},
    reviewer           = {Алексей Недоря}
%   university         = {Санкт-Петербургский Государственный Университет},
%   faculty            = {Математико-механический факультет},
%   city               = {Санкт-Петербург},
%   year               = {2019}
}
% \filltitle{en}{
%     chair              = {Software Engineering},
%     title              = {Automated compiler optimizations tuning},
%     author             = {Alexander Polyakov},
%     supervisorPosition = {},
%     supervisor         = {},
%     reviewerPosition   = {},
%     reviewer           = {},
% }
\maketitle
\tableofcontents
% У введения нет номера главы
\section*{Введение}
Жизнь современного человека невозможно представить без мобильных
устройств, проникших практически во все сферы его жизни. Речь идет о
смартфонах, планшетах, "умных часах" и так далее. При этом человеко-машинное
взаимодействие осуществляется через специальный вид программного обеспечения
--- мобильные приложения. Графический интерфейс является важной компонентой
мобильных приложений, поскольку непосредственно влияет на пользовательский
опыт.

Рынок мобильных устройств и приложений в последние годы непрерывно
растет~\cite{device-market-stat}, также возрастает и разнообразие архитектур
процессоров и операционных систем мобильных устройств ~\cite{cpu-arches, mobile-phones-cpu-trends}.
Это разнообразие, а также необходимость оптимизировать процесс разработки мобильных приложений стали причиной интенсивного развития средств создания
мобильных приложений~\cite{mob-apps-approaches,kotlin-homepage,swiftui-homepage,flutter-homepage, reactnative-homepage, vuenative-homepage}.

На сегодняшний день средства разработки мобильных приложений представляют
собой многокомпонентные программные и/или программно-аппаратные комплексы,
включающие среду разработки на некотором языке программирования, компилятор,
отладчик, окружение исполнения для программ, написанных на этом языке,
подсистему отрисовки графического интерфейса, отладочные платы целевых
устройств и их эмуляторы и так далее.

Несмотря на кажущуюся «обыденность», создание и отрисовка графического
интерфейса приложения являются одними из наиболее сложных задач, с которыми
сталкиваются разработчики мобильных приложений. Использование декларативных
языков для описания интерфейсов и увеличение роли компилятора в процессе
отображения пользовательского интерфейса на экране являются тенденциями
последних лет.

\name{Accord} --- язык программирования общего назначения, сочетающий
элементы компонентного, объектно-ориентированного и функционального
программирования. Данный язык разрабатывается в Санкт-Петербургском
научно-исследовательском центре компании \name{Huawei}. На текущий момент
язык находится в стадии ранней разработки и не обладает устоявшейся спецификацией. Прототип компилятора языка \name{Accord} написан на языке
программирования \name{Go}~\cite{golang-homepage}, имеет автоматически
генерируемый по формальной контекстно-свободной грамматике в расширенной
форме Бэкуса-Наура~\cite{ebnf} и ручной \textit{LL}~\cite{llk-parsers}
парсеры, оптимизатор, работающий над высокоуровневым промежуточным
представлением программы, кодогенерацию в
\name{LLVM IR}~\cite{llvmir-homepage} и байткод некоторой виртуальной
машины. В будущем спецификация языка \name{Accord} и исходный код его
компилятора станут открытыми.

Одно из перспективных направлений применения языка \name{Accord} ---
разработка мобильных приложений. На данный момент, универсальный характер
языка и реализации его компилятора не позволяют оптимально реализовать
необходимую для мобильной разработки функциональность, связанную с
программированием интерфейса пользователя.  Данная работа фокусируется на
внесении изменений в спецификацию языка \name{Accord}, позволяющих
декларативно описывать пользовательский интерфейс приложений, а также на
изменении  его компилятора для предоставления возможности оптимизации 
процесса отображения пользовательского интерфейса с помощью информации о
программе, доступной во время компиляции приложения.
\section{Постановка задачи}
Целью данной работы является внесение изменений в спецификацию языка
\name{Accord}, позволяющих декларативно описывать пользовательский интерфейс
приложений, а также модификация его компилятора, предоставляющая возможность
оптимизации процесса отображения пользовательского интерфейса с помощью
информации о программе, доступной во время компиляции приложения.
Для достижения этой цели были поставлены следующие задачи:
\begin{itemize}
	\item выполнить обзор предметной области и существующих решений;
	\item выполнить сбор и анализ требований к современному языку разработки
	мобильных приложений и его компилятору;
	\item предложить изменения, которые необходимо внести в спецификацию языка
	программирования \name{Accord} и его компилятор для достижения
	поставленной цели;
	\item реализовать данные изменения;
	\item провести апробацию полученного решения.
\end{itemize}
\section{Обзор}
В данном разделе представлен обзор предметной области: преимуществ и
основных способов реализации предметно-ориентированных языков; существующих
языков программирования общего назначения, предоставляющих
пользователям возможность декларативного описания пользовательских интерфейсов с помощью предметно-ориентированных языков; процесс отображения
пользовательских интерфейсов.

\subsection{Предметная область}
\subsubsection{Предметно-ориентированные языки}
Предметно-ориентированный язык (domain-specific language, DSL) -- это язык
программирования с более высоким уровнем абстракции, отражающий специфику
решаемых с его помощью задач.
Такой язык оперирует понятиями и правилами из определенной предметной
области~\cite{book-of-dsls}.

В отличие от языков программирования общего назначения, таких как \name{C},
\name{Python}, \name{Java}, предметно-ориентированные языки предоставляют
абстракции, адекватные решаемой проблеме, позволяя выражать решения,
написанные с их помощью, кратко и ёмко; причём в некоторых случаях
использование DSL не требует квалификации программиста.
В качестве примера DSL можно привести \name{SQL} --  декларативный язык
программирования, применяемый для создания, модификации и управления данными в
реляционной базе данных.
Основным недостатком применения предметно-ориентированных языков является
стоимость их разработки, требующая экспертизы как в области разработки языков
программирования, так и в целевой предметной области.
Это является одной из причин того, что предметные языки редко применяются
для решения задач программной инженерии, в отличие от языков программирования
общего назначения.
Другой причиной отказа от обособленных предметных языков является тот факт,
что сочетание программной библиотеки и языка программирования общего
назначения может заменять DSL.
Программный интерфейс (API) библиотеки содержит специфичный для определённой
области словарь, образованный именами классов, методов и функций, доступный
всем пользователям языков программирования общего назначения, подключившим
библиотеку.
Однако, вышеприведённый подход проигрывает предметным языкам в следующих
аспектах~\cite{when-and-how-develop-dsl,dsl-spectrum-wile}:
\begin{itemize}
	\item устоявшаяся в области нотация, как правило, выходит за рамки
	ограниченных механизмов определения пользовательских операторов,
	предоставляемых языками общего назначения;
	\item абстракции определённой области не всегда могут быть
	просто отображены в конструкции языков общего назначения~\cite{dsl-traversal-transform};
	\item использование предметно-ориентированного языка сохраняет
	возможность анализа, верификации, оптимизации, параллелизации и
	трансформации в рамках конкретной области, что, в случае работы с
	исходным текстом языка программирования общего назначения, является
	более сложной задачей.
\end{itemize}

В последнее время исследования в данной области направлены на категоризацию
предметных языков, а также выработку советов и лучших практик, отвечающих на
вопросы "когда и как?" создавать DSL для конкретной области~\cite{when-and-how-develop-dsl,study-on-preliminary-approaches-develop-dsl,spinellis-dsl-patterns}.

\subsubsection{Подходы к реализации предметно-ориентированных языков}

\subsubsection{Процесс отображения пользовательского интерфейса}


\subsection{Существующие решения}


\section{Архитектура и особенности реализации}


\setmonofont[Mapping=tex-text]{CMU Typewriter Text}
\nocite{*}
\bibliographystyle{ugost2008ls}
\bibliography{diploma.bib}
\end{document}