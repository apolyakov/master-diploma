% Тут используется класс, установленный на сервере Papeeria. На случай, если
% текст понадобится редактировать где-то в другом месте, рядом лежит файл matmex-diploma-custom.cls
% который в момент своего создания был идентичен классу, установленному на сервере.
% Для того, чтобы им воспользоваться, замените matmex-diploma на matmex-diploma-custom
% Если вы работаете исключительно в Papeeria то мы настоятельно рекомендуем пользоваться
% классом matmex-diploma, поскольку он будет автоматически обновляться по мере внесения корректив
%

% По умолчанию используется шрифт 14 размера. Если нужен 12-й шрифт, уберите опцию [14pt]
%\documentclass[14pt]{matmex-diploma}
\documentclass[14pt]{matmex-diploma-custom}
\usepackage{graphicx}
\usepackage{float}
\usepackage{cite}
\usepackage{amsmath,amssymb,amsfonts}
\usepackage{caption}
\usepackage{textcomp}
\usepackage{xcolor}
\usepackage{float}
\usepackage{amssymb}
\usepackage{makecell}
\usepackage[mathscr]{euscript}

\usepackage{listings}
\usepackage{xcolor}

\definecolor{codegreen}{rgb}{0,0.6,0}
\definecolor{codeblue}{rgb}{0,0,0.6}
\definecolor{codegray}{rgb}{0.5,0.5,0.5}
\definecolor{codepurple}{rgb}{0.58,0,0.82}
\definecolor{backcolour}{rgb}{0.95,0.95,0.92}

\lstdefinestyle{mystyle}{
    backgroundcolor=\color{backcolour},   
    commentstyle=\color{codegreen},
    keywordstyle=\color{codeblue},
    numberstyle=\tiny\color{codegray},
    stringstyle=\color{codegreen},
    basicstyle=\ttfamily\footnotesize,
    breakatwhitespace=false,         
    breaklines=true,                 
    captionpos=b,                    
    keepspaces=true,                 
    numbers=left,                    
    numbersep=5pt,                  
    showspaces=false,                
    showstringspaces=false,
    showtabs=false,                  
    tabsize=2
}

\lstdefinelanguage{my_pseudo} {
  morekeywords={function, for, return, let, in, fn, struct, is, var, rx, const, type, this, class},
  sensitive=false,
  morecomment=[l]{//},
  morecomment=[s]{/*}{*/},
  morestring=[b]",
}

% Applies only when you use it
\lstdefinestyle{Antlr}{
    basicstyle=\ttfamily\footnotesize\color{codeblue},%
    breaklines=true,%                                      allow line breaks
    moredelim=[s][\color{green!50!black}\ttfamily]{'}{'},% single quotes in green
    moredelim=*[s][\color{black}\ttfamily]{options}{\}},%  options in black (until trailing })
    commentstyle={\color{gray}\itshape},%                  gray italics for comments
    morecomment=[l]{//},%                                  define // comment
    emph={%
        STRING%                                            literal strings listed here
        },emphstyle={\color{blue}\ttfamily},%              and formatted in blue
    alsoletter={:,|,;},%
    morekeywords={:,|,;},%                                 define the special characters
    keywordstyle={\color{black}},%                         and format them in black
}

\lstset{style=mystyle}
\renewcommand\lstlistingname{Листинг}

\usepackage{indentfirst}



\newcommand\name[1]{\textsc{#1}}

\begin{document}
% Год, город, название университета и факультета предопределены,
% но можно и поменять.
% Если англоязычная титульная страница не нужна, то ее можно просто удалить.
\filltitle{ru}{
    %% Актуально только для курсовых/практик. ВКР защищаются не на кафедре а в ГЭК по направлению, 
    %%   и к моменту защиты вы будете уже не в группе.
    %chair              = {Кафедра, на которой работает научник},
    %group              = {ХХБ.ХХ-мм},
    %% Макрос filltitle ненавидит пустые строки, поэтому обязателен хотя бы символ комментария на строке
    %% Актуально всем.
    title              = {Декларативный предметно-ориентированный язык
    разработки мобильных приложений},
    % 
    %% Здесь указывается тип работы. Возможные значения:
    %%   coursework - отчёт по курсовой работе;
    %%   practice - отчёт по учебной практике;
    %%   prediploma - отчёт по преддипломной практике;
    %%   master - ВКР магистра;
    %%   bachelor - ВКР бакалавра.
    type               = {master},
    author             = {Поляков Александр Романович},
    % 
    %% Актуально только для ВКР. Указывается код и название направления подготовки. Типичные примеры:
    %%   02.03.03 <<Математическое обеспечение и администрирование информационных систем>>
    %%   02.04.03 <<Математическое обеспечение и администрирование информационных систем>>
    %%   09.03.04 <<Программная инженерия>>
    %%   09.04.04 <<Программная инженерия>>
    %% Те, что с 03 в середине --- бакалавриат, с 04 --- магистратура.
    specialty          = {09.04.04 <<Программная инженерия>>},
    % 
    %% Актуально только для ВКР. Указывается шифр и название образовательной программы. Типичные примеры:
    %%   СВ.5006.2017 <<Математическое обеспечение и администрирование информационных систем>>
    %%   СВ.5162.2020 <<Технологии программирования>>
    %%   СВ.5080.2017 <<Программная инженерия>>
    %%   ВМ.5665.2019 <<Математическое обеспечение и администрирование информационных систем>>
    %%   ВМ.5666.2019 <<Программная инженерия>>
    %% Шифр и название программы можно посмотреть в учебном плане, по которому вы учитесь. 
    %% СВ.* --- бакалавриат, ВМ.* --- магистратура. В конце --- год поступления (не обязательно ваш, если вы были в академе/вылетали).
    programme          = {ВМ.5666.2019 <<Программная инженерия>>},
    % 
    %% Актуально только для ВКР, только для матобеса и только 2017-2018 годов поступления. Указывается профиль подготовки, на котором вы учитесь.
    %% Названия профилей можно найти в учебном плане в списке дисциплин по выбору. На каком именно вы, вам должны были сказать после второго курса (можно уточнить в студотделе).
    %% Вот возможные вариканты:
    %%   Математические основы информатики
    %%   Информационные системы и базы данных
    %%   Параллельное программирование
    %%   Системное программирование
    %%   Технология программирования
    %%   Администрирование информационных систем
    %%   Реинжиниринг программного обеспечения
    % profile            = {Системное программирование},
    % 
    %% Актуально всем.
    %supervisorPosition = {проф. каф. СП, д.ф.-м.н., проф.}, % Терехов А.Н.
    supervisorPosition = {к.ф.-м.н., доцент кафедры системного
    программирования}, % Григорьев С.В.   
    supervisor         = {Д.В. Луцив},  
    % 
    %% Актуально только для практик и курсовых. Если консультанта нет, закомментировать или удалить вовсе.
    consultantPosition = {к.ф.-м.н., руководитель направления разработки
    языков программирования\\ООО "Техкомпания Хуавей"},
    consultant         = {А.Е. Недоря},
    %
    %% Актуально только для ВКР.
    reviewerPosition   = {инженер-программист ООО "Техкомпания Хуавей"},
    reviewer           = {Д.И. Соломенников},
}
\maketitle
\tableofcontents
% У введения нет номера главы
\section*{Введение}
Жизнь современного человека невозможно представить без мобильных
устройств, проникших практически во все сферы его жизни. Речь идет о
смартфонах, планшетах, "умных часах" и так далее. При этом человеко-машинное
взаимодействие осуществляется через специальный вид программного обеспечения
--- мобильные приложения. Графический интерфейс является важной компонентой
мобильных приложений, поскольку непосредственно влияет на пользовательский
опыт.

Рынок мобильных устройств и приложений в последние годы непрерывно
растет~\cite{device-market-stat}, также возрастает и разнообразие архитектур
процессоров и операционных систем мобильных устройств ~\cite{cpu-arches, mobile-phones-cpu-trends}.
Это разнообразие, а также необходимость оптимизировать процесс разработки мобильных приложений стали причиной интенсивного развития средств создания
мобильных приложений~\cite{mob-apps-approaches,kotlin-homepage,swiftui-homepage,flutter-homepage, reactnative-homepage, vuenative-homepage}.

На сегодняшний день средства разработки мобильных приложений представляют
собой многокомпонентные программные и/или программно-аппаратные комплексы,
включающие среду разработки на некотором языке программирования, компилятор,
отладчик, окружение исполнения для программ, написанных на этом языке,
подсистему отрисовки графического интерфейса, отладочные платы целевых
устройств и их эмуляторы и так далее.

Несмотря на кажущуюся «обыденность», создание и отрисовка графического
интерфейса приложения являются одними из наиболее сложных задач, с которыми
сталкиваются разработчики мобильных приложений. Использование декларативных
языков для описания интерфейсов и увеличение роли компилятора в процессе
отображения пользовательского интерфейса на экране являются тенденциями
последних лет.

\name{Accord} --- язык программирования общего назначения, сочетающий
элементы компонентного, объектно-ориентированного и функционального
программирования. Данный язык разрабатывается в Санкт-Петербургском
научно-исследовательском центре компании \name{Huawei}. На текущий момент
язык находится в стадии ранней разработки и не обладает устоявшейся спецификацией. Прототип компилятора языка \name{Accord} написан на языке
программирования \name{Go}~\cite{golang-homepage}, имеет автоматически
генерируемый по формальной контекстно-свободной грамматике в расширенной
форме Бэкуса-Наура~\cite{ebnf} и ручной \textit{LL}~\cite{llk-parsers}
парсеры, оптимизатор, работающий над высокоуровневым промежуточным
представлением программы, кодогенерацию в
\name{LLVM IR}~\cite{llvmir-homepage} и байткод некоторой виртуальной
машины. В будущем спецификация языка \name{Accord} и исходный код его
компилятора станут открытыми.

Одно из перспективных направлений применения языка \name{Accord} ---
разработка мобильных приложений. На данный момент, универсальный характер
языка и реализации его компилятора не позволяют оптимально реализовать
необходимую для мобильной разработки функциональность, связанную с
программированием интерфейса пользователя.  Данная работа фокусируется на
внесении изменений в спецификацию языка \name{Accord}, позволяющих
декларативно описывать пользовательский интерфейс приложений, а также на
изменении  его компилятора для предоставления возможности оптимизации 
процесса отображения пользовательского интерфейса с помощью информации о
программе, доступной во время компиляции приложения.
\section{Постановка задачи}
Целью данной работы является внесение изменений в спецификацию языка
\name{Accord}, позволяющих декларативно описывать пользовательский интерфейс
приложений, а также модификация его компилятора, предоставляющая возможность
оптимизации процесса отображения пользовательского интерфейса с помощью
информации о программе, доступной во время компиляции приложения.
Для достижения этой цели были поставлены следующие задачи:
\begin{itemize}
	\item выполнить обзор предметной области и существующих решений;
	\item выполнить сбор и анализ требований к современному языку разработки
	мобильных приложений и его компилятору;
	\item предложить изменения, которые необходимо внести в спецификацию языка
	программирования \name{Accord} и его компилятор для достижения
	поставленной цели;
	\item реализовать данные изменения;
	\item провести апробацию полученного решения.
\end{itemize}
\section{Обзор}
В данном разделе представлен обзор предметной области: преимуществ и
основных способов реализации предметно-ориентированных языков; существующих
языков программирования общего назначения, предоставляющих
пользователям возможность декларативного описания пользовательских интерфейсов с помощью предметно-ориентированных языков; процесс отображения
пользовательских интерфейсов.

\subsection{Предметная область}
\subsubsection{Предметно-ориентированные языки}
Предметно-ориентированный язык (domain-specific language, DSL) -- это язык
программирования с более высоким уровнем абстракции, отражающий специфику
решаемых с его помощью задач.
Такой язык оперирует понятиями и правилами из определенной предметной
области~\cite{book-of-dsls}.

В отличие от языков программирования общего назначения, таких как \name{C},
\name{Python}, \name{Java}, предметно-ориентированные языки предоставляют
абстракции, адекватные решаемой проблеме, позволяя выражать решения,
написанные с их помощью, кратко и ёмко; причём в некоторых случаях
использование DSL не требует квалификации программиста.
В качестве примера DSL можно привести \name{SQL} --  декларативный язык
программирования, применяемый для создания, модификации и управления данными в
реляционной базе данных.
Основным недостатком применения предметно-ориентированных языков является
стоимость их разработки, требующая экспертизы как в области разработки языков
программирования, так и в целевой предметной области.
Это является одной из причин того, что предметные языки редко применяются
для решения задач программной инженерии, в отличие от языков программирования
общего назначения.
Другой причиной отказа от обособленных предметных языков является тот факт,
что сочетание программной библиотеки и языка программирования общего
назначения может заменять DSL.
Программный интерфейс (API) библиотеки содержит специфичный для определённой
области словарь, образованный именами классов, методов и функций, доступный
всем пользователям языков программирования общего назначения, подключившим
библиотеку.
Однако, вышеприведённый подход проигрывает предметным языкам в следующих
аспектах~\cite{when-and-how-develop-dsl,dsl-spectrum-wile}:
\begin{itemize}
	\item устоявшаяся в области нотация, как правило, выходит за рамки
	ограниченных механизмов определения пользовательских операторов,
	предоставляемых языками общего назначения;
	\item абстракции определённой области не всегда могут быть
	просто отображены в конструкции языков общего назначения~\cite{dsl-traversal-transform};
	\item использование предметно-ориентированного языка сохраняет
	возможность анализа, верификации, оптимизации, параллелизации и
	трансформации в рамках конкретной области, что, в случае работы с
	исходным текстом языка программирования общего назначения, является
	более сложной задачей.
\end{itemize}

В последнее время исследования в данной области направлены на категоризацию
предметных языков, а также выработку советов и лучших практик, отвечающих на
вопросы "когда и как?" создавать DSL для конкретной области~\cite{when-and-how-develop-dsl,study-on-preliminary-approaches-develop-dsl,spinellis-dsl-patterns}.

\subsubsection{Подходы к реализации предметно-ориентированных языков}

\subsubsection{Процесс отображения пользовательского интерфейса}


\subsection{Существующие решения}

\section{Требования к спецификации и компилятору языка разработки мобильных приложений}
\label{requirements-section}
В данном разделе представлены требования к современному средству
разработки мобильных приложений, собранные на основе обзора существующих
решений, а также сводная таблица соответствия существующих решений собранным
требованиям. Предъявляемые требования были разделены на две группы:
функциональные и нефункциональные.


\subsection{Функциональные требования}
\subsubsection*{Оптимизация отрисовки графического интерфейса с использованием статической информации о программе}

\subsubsection*{Реактивные обновления пользовательского интерфейса}


\subsection{Нефункциональные требования}
\subsubsection*{Декларативность описания пользовательского интерфейса мобильного приложения}

\subsubsection*{Предоставление отладочных возможностей}

\subsubsection*{Кроссплатформенная разработка}

\subsubsection*{Поддержка интегрированной средой разработки}


\subsection{Соответствие существующих решений собранным требованиям}
\begin{table}[h]
	\begin{tabular}{|c|c|c|c|c|}
		\hline
		
		& \textit{VueJS} & \textit{SwiftUI} &
		\textit{Flutter} & \textit{Flow9} \\
		
		\hline
		\makecell{Оптимизация отрисовки\\графического интерфейса\\
		с использованием\\статической информации\\о программе}
		& -- & + & -- & -- \\
		
		\hline
		\makecell{Реактивные обновления\\пользовательского интерфейса}
		& + & + & -- & + \\
		
		\hline
		\makecell{Декларативность описания\\пользовательского интерфейса}
		& + & + & + & + \\
		
		\hline
		\makecell{Кроссплатформенная разработка}
		& + & -- & + & + \\
		
		\hline
		\makecell{Предоставление отладочных\\возможностей}
		& + & + & + & + \\
		
		\hline
		\makecell{Поддержка интегрированной\\средой разработки}
		& + & + & + & + \\
		
		\hline
        \end{tabular}
        \caption{Сводная таблица соответствия существующих решений собранным требованиям}
        \label{existing-solutions-table}
    \end{table}

Как видно из таблицы~\ref{existing-solutions-table}, на данный момент
наиболее популярные языки разработки мобильных приложений не удовлетворяют
всем требованиям, собранным в данной работе.

\section{Архитектура и особенности реализации}
В данном разделе представлены архитектурные решения разрабатываемого языка и
и некоторые особенности их реализации.

\subsection{Архитектурные решения}
\label{section:architecture}
В ходе данной работы, выбор тех или иных архитектурных решений был
мотивирован удовлетворением итогового результата работы всем собранным
в главе~\ref{requirements-section} требованиям к спецификации и компилятору
современного языка разработки мобильных приложений.

\subsubsection{Встраивание  предметного языка в язык программирования \name{Accord}}
Язык разработки мобильных приложений является примером
предметно-ориентированного языка. В качестве способа реализации такого языка
был выбран метод встраивания предметного языка в базовый, которым является
язык \name{Accord}. Такое решение позволило переиспользовать существующий
прототип компилятора и спецификацию языка \name{Accord}, включая синтаксис,
синтаксический и семантический анализаторы, оптимизатор высокоуровневого
промежуточного представления, кодогенерации в \name{LLVM IR} и байткод
некоторой виртуальной машины.

Следствием этого решения является тот факт, что все синтаксические и
семантические конструкции и правила являются общими как для разработки
мобильных интерфейсов, так и для программ общего назначения. Другим
следствием является автоматическое соответствие решения таким требованиям,
как:
\begin{itemize}
	\item кроссплатформенная разработка: наличие кодогенерации в
	\name{LLVM IR} означает возможность потенциального запуска мобильных
	приложений, написанных на языке \name{Accord}, на всех платформах,
	поддерживаемых проектом \name{LLVM}~\cite{llvm-homepage}, а также
	на платформах, поддерживаемых виртуальной машиной, в байткод которой
	может быть оттранслирован исходный код приложения;
	\item предоставление отладочных возможностей: тот факт, что язык
	разработки мобильных приложений встроен в язык \name{Accord}, что
	означает их единство, гарантирует предоставление отладочных возможностей
	при разработке мобильных приложений в случае, если язык \name{Accord}
	имеет эти возможности. Несмотря на раннюю стадию разработки, язык
	\name{Accord} уже способен сохранять отладочную информацию о программе
	в формате \name{DWARF}~\cite{dwarf-homepage};
	\item поддержка интегрированной средой разработки: любая интегрированная
	среда разработки для языка \name{Accord} подходит и для разработки
	мобильных приложений на нём в силу единства языка разработки мобильных
	приложения и языка \name{Accord}.
\end{itemize}

Для достижения декларативности описания графического интерфейса в
спецификацию языка \name{Accord} были добавлены процедуры инициализации
объектов. На листинге~\ref{lst:accord-init} представлено упрощённое
синтаксическое правило определения процедур инициализации внутри
синтаксического контекста определения типа. Вызов процедуры инициализации
происходит автоматически при создании объекта во время выполнения программы.
Синтаксически данная семантика выглядит следующим образом:
\textit{TypeName(args)}, например, \textit{Text("This is an example")}.
\begin{lstlisting}[style=Antlr, caption=Синтаксическое правило процедур инициализации, label={lst:accord-init}]
INIT
	: 'init'
	;

type_init
	: INIT fn_params expression_sequence
	;
\end{lstlisting}

Для работы с реактивными данным в спецификацию языка \name{Accord} был
добавлен модификатор полей типов --- \textit{rx}. На данный момент, для
данных графической компоненты, помеченных модификатором \textit{rx},
компилятор автоматически генерирует функцию-мутатор. Задачей данной функции
является изменение реактивных данных и отметка необходимости обновления
графической компоненты, реактивные данные которой были изменены. В ходе
анализа графа потока управления, все изменения реактивных данных заменяются
на вызов сгенерированной функции-мутатора.
На листинге~\ref{lst:accord-rx-setter} представлен пример мутатора
реактивного поля \textit{counter} типа \textit{i32} компоненты
\textit{CounterComponent}.
\begin{lstlisting}[language=my_pseudo, caption=Пример функции-мутатора реактивных данных, label={lst:accord-rx-setter}]
fn CounterComponent.aco.set_counter(val: i32) {
    counter = val
    markNeedUpdate(this)
}
\end{lstlisting}
В дальнейшем семантика модификатора \textit{rx} будет расширена: он будет
применим не только к полям типов, но и к переменным. Изменение реактивных
данных будет приводить не только к обновлению графической комопненты на
экране, но и к обновлению других данных, как-либо использующих изменившиеся
реактивные данные.

\subsubsection{Статическая типизация графических компонент}
Различные графические компоненты могут сильно отличаться друг друга
семантически. Так, одни компоненты имеют динамическую
природу и могут изменяться от кадра к кадру, другие же --- статические ---
создаются лишь раз и не меняются на протяжении всей работы приложения.
Реальные графические компоненты, определяемые пользователем в приложении
могут быть достаточно сложными: включать большое количество различных
компонент, каждая из которых имеет своё собственное поведение и условия
обновления. Классический алгоритм обновления кадра, описанный
в главе~\ref{section:render-pipeline}, при условии поддержки компилятором,
использует статическую информацию о компонентах для минимизации
количества действий, необходимых для обновления кадра. Так, информация о
том, может ли компонента изменяться во время работы приложения, используется
средой времени исполнения для уменьшения количества сравнений
соответствующих компонент двух соседних кадров. Однако, знание таких
параметров, как тип и размер компоненты и её подкомпонент, а также
расположение подкомпонент относительно их родительской компоненты позволяет
избавиться от части операций, проводимых классическим алгоритмом во время
исполнения приложения ещё на этапе компиляции программы.

Рассмотрим пример графической компоненты на
листинге~\ref{lst:component-example}. Она представляет собой колонку,
состоящую из текста, кнопки и некоторой условной компоненты, которая
превращается в изображение или текст в зависимости от условия
\textit{condition}.
\begin{lstlisting}[language=my_pseudo, caption=Пример графической компоненты, label={lst:component-example}]
Column {
    Text("Current count: ${counter}")
    Button("Click on me!")
        .onClick(fn() { counter += 1 })
        .backgroundColor(Color.Green)
        .width(150px)
    ConditionalView(
        condition,
        Image("img.png"),
        Text("Empty")
    )
}
\end{lstlisting}
Зная статический тип данной компоненты
(листинг~\ref{lst:component-type-example}: $f0$ --- тип
под\-ком\-по\-ненты-колонки, $f1$ --- тип переменной \textit{counter}, $f2$
--- тип переменной \textit{condition}) и её представление в памяти
(рис.~\ref{component-layout-example}), компилятор способен ещё во время
компиляции приложения понять, какие компоненты могут быть изменены
во время работы приложения, а какие нет.
\begin{lstlisting}[escapeinside={(*}{*)}, caption=Пример статического типа компоненты, label={lst:component-type-example}]
struct {
    f0: Column[Text, Button, ConditionalView[Image, Text]]
    f1: i32
    f2: bool
}
\end{lstlisting}

\begin{figure}[H]
\centering
\MemoryLayout{
        32/blue!40/f0,
        40/green!40/f1,
        48/red!40/f2
}
\caption{Пример представления компоненты в памяти}
\label{component-layout-example}
\end{figure}
Имея представление компоненты в
памяти, компилятор может сгенерировать специализированную для конкретного
типа компоненты процедуру обновления
(листинг~\ref{lst:optimized-update-fn}). Эта процедура состоит из точечных
вызовов обновления только потенциально изменяемых во время работы приложения
компонент. Для сравнения такого подхода с классическим, введём понятия
некоторых условных операций, необходимых для обновления интерфейса согласно
главе~\ref{section:render-pipeline}. Пусть $A$ --- операция перехода между
узлами дерева компонент, $B$ --- проверка узла на изменяемость (проверка во
время исполнения программы), $C$ --- вызов процедуры обновления компоненты.
Тогда, если $N$ --- количество всех компонент, $M$ --- количество изменяемых
компонент, причём $M \leq N$, то для обновления одного кадра классическому
алгоритму необходимо произвести $(N - 1) * A + N * B + M * C$ операций,
в то время как описанному выше алгоритму лишь $M * C$.
\begin{lstlisting}[language=my_pseudo, caption=Пример сгенерированной процедуры обновления компоненты, label={lst:optimized-update-fn}]
fn Counter.rerender() {
    // skip Column
    Text.rerender(fieldAddress(0, 0), args...)
    // skip Button
    ConditionalView.rerender(
        fieldAddress(0, 2), args...,
    )
}
\end{lstlisting}
Для того, чтобы данная информация была доступна компилятору языка
\name{Accord}, в его спецификацию были добавлены синтаксис и семантика
наследования независимых типов (не имеющих типов-параметров) от
ненастроенных обобщённых типов с последующей автоматической настройкой
обобщённого типа-родителя в зависимости от содержимого определения
типа-потомка.

\subsection{Вывод}
Результатом выбора и реализации архитектурных решений, описанных в
пункте~\ref{section:architecture}, стало соответствие разработанного
решения всем требованиям, перечисленным в главе~\ref{requirements-section}.

\section{Результаты}
На данный момент в рамках преддипломной научной практики были достигнуты
следующие результаты:
\begin{itemize}
	\item Выполнен обзор предметной
		\begin{itemize}
			\item Предметно-ориентированные языки
			\item Отрисовка графического интерфейса
		\end{itemize}
	\item Выполнен обзор существующих решений
		\begin{itemize}
			\item \textit{Dart/Flutter}
			\item \textit{Kotlin UI DSL}
			\item \textit{React Native}
			\item \textit{Swift/SwiftUI}
		\end{itemize}
	\item Собраны требования к современному языку разработки мобильных
	приложений и его компилятору
	\item Предложены изменения спецификации и компилятора языка
	\textit{Accord}, позволяющие языку \textit{Accord} удовлетворить
	всем собранным требованиям
	\item Предложенные изменения реализованы в прототипе компилятора
	языка \textit{Accord}
\end{itemize}

Ближайшими планами является проведение апробации полученного решения.

\begin{appendices}
\section{Рекомендации по выбору подхода \newline  к созданию предметно-ориентирован-\\ного языка}
\label{appendixA}

\begin{figure}[h]
\label{dsl-impl-approach-guideline}
\centering
\includegraphics[width=\linewidth,height=1.1\linewidth,keepaspectratio]{resources/dsl-implementation-guideline.png}
\caption{Алгоритм выбора метода разработки DSL~\cite{when-and-how-develop-dsl}}
\end{figure}

\end{appendices}



\setmonofont[Mapping=tex-text]{CMU Typewriter Text}
\nocite{*}
\bibliographystyle{ugost2008ls}
\bibliography{diploma.bib}
\end{document}