% Тут используется класс, установленный на сервере Papeeria. На случай, если
% текст понадобится редактировать где-то в другом месте, рядом лежит файл matmex-diploma-custom.cls
% который в момент своего создания был идентичен классу, установленному на сервере.
% Для того, чтобы им воспользоваться, замените matmex-diploma на matmex-diploma-custom
% Если вы работаете исключительно в Papeeria то мы настоятельно рекомендуем пользоваться
% классом matmex-diploma, поскольку он будет автоматически обновляться по мере внесения корректив
%

% По умолчанию используется шрифт 14 размера. Если нужен 12-й шрифт, уберите опцию [14pt]
%\documentclass[14pt]{matmex-diploma}
\documentclass[14pt]{matmex-diploma-custom}
\usepackage{graphicx}
\usepackage{float}
\usepackage{cite}
\usepackage{amsmath,amssymb,amsfonts}
\usepackage{caption}
\usepackage{textcomp}
\usepackage{xcolor}
\usepackage{float}
\usepackage{amssymb}
\usepackage[mathscr]{euscript}

\usepackage{indentfirst}



\newcommand\name[1]{\textsc{#1}}

\begin{document}
% Год, город, название университета и факультета предопределены,
% но можно и поменять.
% Если англоязычная титульная страница не нужна, то ее можно просто удалить.
\filltitle{ru}{
    chair              = {Программная инженерия},
    title              = {Декларативный предметно-ориентированный язык разработки интерфейсов мобильных приложений},
    type               = {coursework},
    position           = {студента},
    group              = 671,
    author             = {Поляков Александр Романович},
    supervisorPosition = {старший преподаватель},
    supervisor         = {Дмитрий Луцив},
    reviewerPosition   = {исследователь в ООО Хуавей},
    reviewer           = {Алексей Недоря}
%   university         = {Санкт-Петербургский Государственный Университет},
%   faculty            = {Математико-механический факультет},
%   city               = {Санкт-Петербург},
%   year               = {2019}
}
% \filltitle{en}{
%     chair              = {Software Engineering},
%     title              = {Automated compiler optimizations tuning},
%     author             = {Alexander Polyakov},
%     supervisorPosition = {},
%     supervisor         = {},
%     reviewerPosition   = {},
%     reviewer           = {},
% }
\maketitle
\tableofcontents
% У введения нет номера главы
\section*{Введение}
Жизнь человека в настоящее время тяжело представить без носимых устройств,
проникших практически во все сферы деятельности оного.
Человеко-машинное взаимодействие в данном случае в большинстве своём
осуществляется посредством мобильных приложений -- программного обеспечения, специально разработанного для запуска на мобильных устройствах, таких как
смартфоны, планшеты, умные часы, автомобили.
Изобилие устройств повлекло за собой разнообразие архитектур
процессоров~\cite{cpu-arches, mobile-phones-cpu-trends} и операционных систем, которые необходимо учитывать при разработке приложения.

Несмотря на то, что в создании современных мобильных приложений явно
прослеживается тренд на унификацию методологий, интерфейсов, компонентов и других атрибутов разработки программного обеспечения, своеобразными
аттракторами данной унификации стали пара наиболее популярных мобильных операционных систем вкупе с несколькими схожими архитектурами процессоров. 
Перенос программного обеспечения на другие платформы до сих пор остаётся одним
из основных подходов к разработке мультиплатформенных мобильных
приложений~\cite{mob-apps-approaches}.
В последнее время набирают популярность средства разработки программного
обеспечения~\cite{kotlin-homepage,swift-homepage,flutter-homepage,
reactnative-homepage, vuenative-homepage}, позволяющие разработчикам работать
над единой кодовой базой приложения, предназначенной для работы с несколькими
конфигурациями пользовательских устройств.

Одним из важных требований к современным средствам разработки
мультиплатформенных мобильных приложений является возможность декларативного
описания пользовательского интерфейса.
Такая возможность позволяет ускорить и удешевить разработку мобильных
приложений за счёт разделения труда между программистами логики приложения и
дизайнерами пользовательского интерфейса, сохранив при этом единство окружения
разработки и исполнения.
Современным и популярным подходом к предоставлению пользователям данной
функциональности является использование декларативных
предметно-ориентированных языков -- языков программирования с высоким уровнем
абстракции, отражающих специфику решаемых с их помощью задач, оперируя понятиями и правилами из определённой области.

\name{Accord} -- язык программирования общего назначения, зародившийся и
разрабатываемый в российском научно-исследовательском институте компании
\name{Huawei} и находящийся на данный момент в стадии ранней разработки.
Одной из наиболее перспективных ниш для данного языка является разработка
мобильных приложений, требующая от языка, как было описано выше, возможности
декларативного описания интерфейсов мультиплатформенных приложений.

\section{Постановка задачи}
\section{Постановка задачи}
Целью данной работы является создание языка разработки
мобильных приложений, основанного на языке программирования общего
назначения \name{Accord}. Для её достижения были поставлены следующие задачи:
\begin{itemize}
	\item выполнить сбор и анализ требований к современному языку
	разработки мобильных приложений;
	\item выполнить обзор предметной области и существующих решений;
	\item предложить подход к созданию языка разработки мобильных
	приложений, удовлетворяющему собранным требованиям;
	\item реализовать данный язык;
	\item провести его апробацию.
\end{itemize}

\section{Обзор}
\subsection{Обзор предметной области}
\subsection{Обзор существующих решений}

\section{Архитектура и особенности реализации}


\setmonofont[Mapping=tex-text]{CMU Typewriter Text}
\nocite{*}
\bibliographystyle{ugost2008ls}
\bibliography{diploma.bib}
\end{document}