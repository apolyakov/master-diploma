\section{Архитектура и особенности реализации}
В данном разделе представлена общая архитектура решения и особенности реализации, позволившие решению удовлетворить всем требованиям к
современному языку разработки мобильных приложений, указанным в разделе~\ref{requirements-section}.

\subsection{Архитектура решения}

\subsection{Особенности реализации}
\subsubsection*{Встраивание предметного языка разработки мобильных приложений в язык программирования \textit{Accord}}
Реализация предметно-ориентированного языка разработки мобильных приложений
методом встраивания данного предметного языка в базовый язык, которым в
данном случае выступает язык программирования \textit{Accord} сохраняет
преимущества и недостатки данного подхода, описанные в разделе~\ref{dsl-section}.
Однако, предметно-специфичные анализ, оптимизации и трансформации были
учтены на этапе проектирования данного решения, что нивелировало часть
недостатков подхода.

Основная функциональность языка \textit{Accord}, использованная для
создания предметного языка разработки приложений:
\begin{itemize}
	\item Легковесные структуры
	\item Статическая типизация
	\item Семантический анализ
	\item Реактивность на уровне языка
	\item Кроссплатформенность
\end{itemize}


\subsubsection*{Разделение дерева компонент на статическую и динамическую части}
Благодаря статической типизации, на стадии компиляции возможно разделить
пользовательский интерфейс, представленный деревом компонент,
на две обособленные в памяти приложения составляющие.
\begin{itemize}
	\item Статическая часть дерева: отсутствие изменения структуры
	дерева компонент во время исполнения мобильного приложения позволяет
	полностью исключить узлы такого дерева из дорогостоящего алгоритма
	нахождения разницы между деревьями элементов интерфейса.
	\item Динамическая часть дерева: знание о том, какие элементы
	интерфейса могут стать точкой начала изменения структуры дерева
	компонент позволяет компилятору вызывать алгоритм нахождения разницы
	между деревьями элементов точечно, минимизируя количество входных
	данных этого алгоритма.
\end{itemize}


\subsubsection*{Выделение данных состояния приложения в памяти}


\subsubsection*{Декларативность}