\section*{Введение}
Жизнь человека в настоящее время невозможно представить без мобильных
устройств, проникших практически во все сферы его жизни. Речь идет о
смартфонах, планшетах, умных часах и так далее. При этом человеко-машинное
взаимодействие осуществляется через специальный вид программного обеспечения --- мобильные приложения. Графический интерфейс является важной компонентой
мобильных приложений, поскольку непосредственно влияет на пользовательский
опыт.

Рынок мобильных устройств и приложений в последние годы непрерывно
растет~\cite{device-market-stat}, что является причиной увеличения разнообразия и сложности архитектур процессоров и операционных систем
мобильных устройств~\cite{cpu-arches, mobile-phones-cpu-trends}. Это
разнообразие, а также необходимость оптимизировать процесс разработки мобильных приложений стали причиной интенсивного развития средств создания
мобильных приложений~\cite{mob-apps-approaches,kotlin-homepage,swiftui-homepage,flutter-homepage, reactnative-homepage, vuenative-homepage}.

Конкуренция и технический прогресс постоянно расширяют и усложняют
требования, предъявляемые к средствам разработки мобильных приложений. На
сегодняшний день эти средства представляют собой многокомпонентные программные или программно-аппаратные комплексы, включающие среду разработки
на некотором языке программирования, компилятор, отладчик и окружение исполнения для программ, написанных на этом языке, подсистему отрисовки
графического интерфейса, отладочные платы целевых устройств и их эмуляторы.
Таким образом, создавая новое средство разработки мобильных приложений,
необходимо спроектировать и создать вышеописанные компоненты с тем, чтобы
итоговое средство отвечало как можно большему числу предъявляемых требований.

Создание и отрисовка графического интерфейса приложения являются одними из
наиболее сложных задач [X], с которыми сталкиваются разработчики мобильных
приложений. Использование декларативных языков для описания интерфейсов и
увеличение роли статической компиляции в процессе отображения
пользовательского интерфейса на экране являются тенденциями последних лет.

\name{Accord} --- язык программирования общего назначения, сочетающий
элементы компонентного, объектно-ориентированного и функционального
программирования. Данный язык разрабатывается в Санкт-Петербургском
научно-исследовательском центре компании \name{Huawei}. На текущий момент
язык находится в стадии ранней разработки и не обладает устоявшейся спецификацией. Прототип компилятора языка \name{Accord} написан на языке
программирования \name{Go} [X], имеет автоматически генерируемый по
формальной контекстно-свободной грамматике в расширенной форме Бэкуса-Наура
[X] и ручной LL(1) парсеры, оптимизатор, работающий над высокоуровневым
промежуточным представлением программы, кодогенерацию в \name{LLVM IR} [X] и
байткод некоторой виртуальной машины. В будущем спецификация языка
\name{Accord} и исходный код его компилятора станут открытыми.

Одно из перспективных направлений применения языка \name{Accord} ---
разработка мобильных приложений. Данная работа фокусируется на внесении
изменений в спецификацию языка \name{Accord} и его компилятор с тем, чтобы
данный язык мог использоваться в качестве предметно-ориентированного языка
разработки мобильных приложений, учитывающего современные тенденции данной 
области.