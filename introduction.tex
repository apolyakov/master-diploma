\section*{Введение}
Жизнь современного человека невозможно представить без мобильных
устройств, проникших практически во все сферы его жизни. Речь идет о
смартфонах, планшетах, "умных часах" и так далее. При этом человеко-машинное
взаимодействие осуществляется через специальный вид программного обеспечения
--- мобильные приложения. Графический интерфейс является важной компонентой
мобильных приложений, поскольку непосредственно влияет на их доступность и
удобство использования.

Рынок мобильных устройств и приложений в последние годы непрерывно
растет~\cite{device-market-stat,app-download-stat}, также возрастает и
разнообразие архитектур процессоров и операционных систем мобильных
устройств ~\cite{cpu-arches, mobile-phones-cpu-trends}.
Это разнообразие, а также необходимость оптимизировать процесс разработки
мобильных приложений стали причиной интенсивного развития средств создания
мобильных приложений~\cite{mob-apps-approaches,jp-compose-homepage,swiftui-homepage,flutter-homepage, reactnative-homepage, vuenative-homepage}.

На сегодняшний день средства разработки мобильных приложений представляют
собой многокомпонентные программные и/или про\-грамм\-но-аппаратные
комплексы, включающие среду разработки на некотором языке программирования,
компилятор, отладчик, среду времени выполнения для программ, написанных на
этом языке, подсистему отрисовки графического интерфейса, отладочные платы
целевых ус\-т\-рой\-ств и их эмуляторы и так далее.

Несмотря на кажущуюся «обыденность», создание и отрисовка графического
интерфейса приложения являются одними из наиболее сложных задач, с которыми
сталкиваются разработчики мобильных приложений. Использование декларативных
языков для описания интерфейсов и увеличение роли компилятора в процессе
отображения пользовательского интерфейса на экране являются тенденциями
последних лет в данной области.

\name{Accord} --- язык программирования общего назначения, сочетающий
элементы компонентного, объектно-ориентированного и функционального
программирования. Данный язык разрабатывается в Санкт-Петербургском
научно-исследовательском центре компании \name{Huawei}. На текущий момент
язык находится в стадии ранней разработки и не обладает устоявшейся
спецификацией. Прототип компилятора языка написан на языке
программирования \name{Go}~\cite{golang-homepage}, имеет автоматически
генерируемый по формальной контекстно-свободной грамматике в расширенной
форме Бэкуса-Наура~\cite{ebnf} и ручной \textit{LL}~\cite{llk-parsers}
парсеры, оптимизатор, работающий над высокоуровневым промежуточным
представлением программы, кодогенерацию в
\name{LLVM IR}~\cite{llvmir-homepage} и байткод некоторой виртуальной
машины. В будущем спецификация языка \name{Accord} и исходный код его
компилятора станут открытыми.

Одно из перспективных направлений применения языка \name{Accord} ---
разработка мобильных приложений. На данный момент, универсальный характер
языка и реализации его компилятора не позволяют оптимально реализовать
необходимую для мобильной разработки функциональность, связанную с
программированием интерфейса пользователя.  Данная работа фокусируется на
внесении изменений в спецификацию и компилятор языка \name{Accord},
позволяющих использовать его в качестве языка разработки мобильных
приложений, не уступающего существующим аналогичным решениям.
