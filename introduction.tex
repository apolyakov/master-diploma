\section*{Введение}
Жизнь человека в настоящее время тяжело представить без мобильных устройств,
проникших практически во все сферы его жизни.
К ним относятся смартфоны, планшеты, умные часы и так далее.
Человеко-машинное взаимодействие при этом осуществляется через мобильные
приложения --- программное обеспечение, специально разработанное для
запуска на мобильных устройствах. Графический интерфейс является
компонентой приложения, которая напрямую влияет на пользовательский опыт.

Размер рынка мобильных устройств~\cite{device-market-stat} и
приложений стал причиной увеличения количества используемых
в данной области архитектур
процессоров~\cite{cpu-arches, mobile-phones-cpu-trends} и
операционных систем. Это разнообразие, а также желание бизнеса
оптимизировать процесс разработки мобильных приложений
стали причиной интенсивного развития средств создания мобильных
приложений~\cite{mob-apps-approaches,kotlin-homepage,swift-homepage,flutter-homepage, reactnative-homepage, vuenative-homepage}.

Высокая конкуренция и технический прогресс постоянно добавляют
и усиливают требования, предъявляемые к средствам разработки мобильных
приложений, которые на сегодняшний день представляют собой
многокомпонентные программные или программно-аппарат\-ные комплексы,
включающие среду разработки на некотором языке программирования,
компилятор, отладчик и окружение исполнения этого языка, подсистему
отрисовки графического интерфейса, отладочные платы целевых устройств
или их эмуляторы.
Таким образом, создавая новое средство разработки мобильных
приложений, необходимо спроектировать и создать вышеописанные
компоненты с тем, чтобы их композиция отвечала как можно большему
числу предъявляемых требований.

Создание и отрисовка графического интерфейса приложения являются одними
из наиболее сложных задач, с которыми сталкиваются разработчики
мобильных приложений. Использование декларативных языков для описания
интерфейсов и увеличение роли статической компиляции в процессе отображения
пользовательского интерфейса на экране являются одними из прослеживающихся
в данной области тенденций.

\name{Accord} --- язык программирования общего назначения, сочетающий
элементы компонентного, объектно-ориентированного и функционального
программирования. Данный язык разрабатывается в российском
научно-исследовательском институте компании \name{Huawei}. На текущий момент
язык находится в стадии ранней разработки. Одно из перспективных
направлений применения данного языка в будущем --- разработка мобильных
приложений. Данная работа фокусируется на разработке языка программирования
\name{Accord} и его компилятора с учётом вышеобозначенных тенденций
в рассматриваемой области.
