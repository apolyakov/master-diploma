\section*{Введение}
Жизнь человека в настоящее время тяжело представить без носимых устройств,
проникших практически во все сферы деятельности оного.
Человеко-машинное взаимодействие в данном случае в большинстве своём
осуществляется посредством мобильных приложений -- программного обеспечения, специально разработанного для запуска на мобильных устройствах, таких как
смартфоны, планшеты, умные часы, автомобили.
Изобилие устройств повлекло за собой разнообразие архитектур
процессоров~\cite{cpu-arches, mobile-phones-cpu-trends} и операционных систем, которые необходимо учитывать при разработке приложения.

Несмотря на то, что в создании современных мобильных приложений явно
прослеживается тренд на унификацию методологий, интерфейсов, компонентов и других атрибутов разработки программного обеспечения, своеобразными
аттракторами данной унификации стали пара наиболее популярных мобильных операционных систем вкупе с несколькими схожими архитектурами процессоров. 
Перенос программного обеспечения на другие платформы до сих пор остаётся одним
из основных подходов к разработке мультиплатформенных мобильных
приложений~\cite{mob-apps-approaches}.
В последнее время набирают популярность средства разработки программного
обеспечения~\cite{kotlin-homepage,swift-homepage,flutter-homepage,
reactnative-homepage, vuenative-homepage}, позволяющие разработчикам работать
над единой кодовой базой приложения, предназначенной для работы на нескольких
конфигурациях пользовательских устройств.

Одним из важных требований к современным средствам разработки
мультиплатформенных мобильных приложений является возможность декларативного
описания пользовательского интерфейса.
Такая возможность позволяет ускорить и удешевить разработку мобильных
приложений за счёт разделения труда между программистами логики приложения и
дизайнерами пользовательского интерфейса, сохранив при этом единство окружения
разработки и исполнения.
Современным и популярным подходом к предоставлению пользователям данной
функциональности является использование декларативных
предметно-ориентированных языков -- языков программирования с высоким уровнем
абстракции, отражающих специфику решаемых с их помощью задач, оперируя понятиями и правилами из определённой области.

\name{Accord} -- язык программирования общего назначения, зародившийся и
разрабатываемый в российском научно-исследовательском институте компании
\name{Huawei} и находящийся на данный момент в стадии ранней разработки.
Одной из наиболее перспективных ниш для данного языка является разработка
мобильных приложений, требующая от языка, как было описано выше, возможности
декларативного описания интерфейсов мультиплатформенных приложений.