\section*{Введение}
Жизнь человека в настоящее время тяжело представить без носимых устройств,
проникших практически во все сферы деятельности оного.
Человеко-машинное взаимодействие в данном случае в большинстве своём
осуществляется посредством мобильных приложений --- программного
обеспечения, специально разработанного для запуска на мобильных устройствах,
таких как смартфоны, планшеты, умные часы.

Огромный размер рынка мобильных устройств~\cite{device-market-stat} и
приложений повлёк за собой увеличение количества используемых в данной области
архитектур процессоров~\cite{cpu-arches, mobile-phones-cpu-trends} и
операционных систем. Подобное разнообразие, а также желание бизнеса
оптимизировать процесс разработки мобильных приложений, в свою очередь,
стали причиной роста количества средств создания мобильных
приложений, предоставляющих своим пользователям функциональность,
упрощающую разработку приложений под конкретную платформу и, в последнее
время, разработку кроссплатформенных приложений~\cite{mob-apps-approaches,kotlin-homepage,swift-homepage,flutter-homepage,
reactnative-homepage, vuenative-homepage}.

Высокая конкуренция и технический прогресс неизменно добавляют
и усиливают требования, предъявляемые к средствам разработки мобильных
приложений текущего и следующего поколений.

Современное средство разработки мобильных приложений представляет собой
многокомпонентный программный или программно-аппарат\-ный комплекс, состоящий
из языка программирования, его компилятора, отладчика, интегрированной
среды разработки, окружения исполнения, отладочных плат целевых устройств
или их эмуляторов и прочих инструментов разработки программного обеспечения.
Таким образом, занимаясь созданием нового средства разработки мобильных
приложений, его авторам необходимо спроектировать и создать вышеописанные
компоненты таким образом, чтобы их композиция отвечала как можно большему
числу требований.


%Изобилие носимых устройств повлекло за собой разнообразие архитектур
%процессоров~\cite{cpu-arches, mobile-phones-cpu-trends} и операционных %систем, которые необходимо учитывать при разработке приложения.

%Несмотря на то, что в создании современных мобильных приложений явно
%прослеживается тренд на унификацию методологий, интерфейсов, компонентов и %других атрибутов разработки программного обеспечения, своеобразными
%аттракторами данной унификации стали пара наиболее популярных мобильных %операционных систем вкупе с несколькими схожими архитектурами процессоров. 
%Перенос программного обеспечения на другие платформы до сих пор остаётся %одним
%из основных подходов к разработке мультиплатформенных мобильных
%приложений~\cite{mob-apps-approaches}.
%В последнее время набирают популярность средства разработки программного
%обеспечения~\cite{kotlin-homepage,swift-homepage,flutter-homepage,
%reactnative-homepage, vuenative-homepage}, позволяющие разработчикам %работать
%над единой кодовой базой приложения, предназначенной для работы с %несколькими
%конфигурациями пользовательских устройств.

%Одним из важных требований к современным средствам разработки
%мультиплатформенных мобильных приложений является возможность декларативного
%описания пользовательского интерфейса.
%Такая возможность позволяет ускорить и удешевить разработку мобильных
%приложений за счёт разделения труда между программистами логики приложения и
%дизайнерами пользовательского интерфейса, сохранив при этом единство окружения
%разработки и исполнения.
%Современным и популярным подходом к предоставлению пользователям данной
%функциональности является использование декларативных
%предметно-ориентированных языков --- языков программирования с высоким %уровнем
%абстракции, отражающих специфику решаемых с их помощью задач, оперируя понятиями и правилами из определённой области.

\name{Accord} --- язык программирования общего назначения, вбирающий в себя
элементы объектно-ориентированного, функционального и компонентного
программирования, зародившийся и разрабатываемый в российском
научно-исследовательском институте компании \name{Huawei}. На текущий момент
язык находится в стадии ранней разработки. Одно из перспективных
направлений применения данного языка в будущем --- разработка мобильных приложений, что требует проектирования полноценного средства разработки
приложений, на двух компонентах которого фокусируется данная работа: языке
программирования \name{Accord} и его компиляторе. Эти компоненты в
дальнейшем будем объединять термином "язык разработки приложений".
Оставшиеся компоненты, входящие в состав средства разработки мобильных приложений, в работе затрагиваются в меньшей степени.