\section{Обзор}
В данном разделе представлен обзор предметной области: преимуществ и
основных способов реализации предметно-ориентированных языков; существующих
языков программирования общего назначения, предоставляющих
пользователям возможность декларативного описания пользовательских интерфейсов с помощью предметно-ориентированных языков; процесс отображения
пользовательских интерфейсов.

\subsection{Предметная область}
\subsubsection{Предметно-ориентированные языки}
Предметно-ориентированный язык (domain-specific language, DSL) -- это язык
программирования с более высоким уровнем абстракции, отражающий специфику
решаемых с его помощью задач.
Такой язык оперирует понятиями и правилами из определенной предметной
области~\cite{book-of-dsls}.

В отличие от языков программирования общего назначения, таких как \name{C},
\name{Python}, \name{Java}, предметно-ориентированные языки предоставляют
абстракции, адекватные решаемой проблеме, позволяя выражать решения,
написанные с их помощью, кратко и ёмко; причём в некоторых случаях
использование DSL не требует квалификации программиста.
В качестве примера DSL можно привести \name{SQL} --  декларативный язык
программирования, применяемый для создания, модификации и управления данными в
реляционной базе данных.
Основным недостатком применения предметно-ориентированных языков является
стоимость их разработки, требующая экспертизы как в области разработки языков
программирования, так и в целевой предметной области.
Это является одной из причин того, что предметные языки редко применяются
для решения задач программной инженерии, в отличие от языков программирования
общего назначения.
Другой причиной отказа от обособленных предметных языков является тот факт,
что сочетание программной библиотеки и языка программирования общего
назначения может заменять DSL.
Программный интерфейс (API) библиотеки содержит специфичный для определённой
области словарь, образованный именами классов, методов и функций, доступный
всем пользователям языков программирования общего назначения, подключившим
библиотеку.
Однако, вышеприведённый подход проигрывает предметным языкам в следующих
аспектах~\cite{when-and-how-develop-dsl,dsl-spectrum-wile}:
\begin{itemize}
	\item устоявшаяся в области нотация, как правило, выходит за рамки
	ограниченных механизмов определения пользовательских операторов,
	предоставляемых языками общего назначения;
	\item абстракции определённой области не всегда могут быть
	просто отображены в конструкции языков общего назначения~\cite{dsl-traversal-transform};
	\item использование предметно-ориентированного языка сохраняет
	возможность анализа, верификации, оптимизации, параллелизации и
	трансформации в рамках конкретной области, что, в случае работы с
	исходным текстом языка программирования общего назначения, является
	более сложной задачей.
\end{itemize}

В последнее время исследования в данной области направлены на категоризацию
предметных языков, а также выработку советов и лучших практик, отвечающих на
вопросы "когда и как?" создавать DSL для конкретной области~\cite{when-and-how-develop-dsl,study-on-preliminary-approaches-develop-dsl,spinellis-dsl-patterns}.

\subsubsection{Подходы к реализации предметно-ориентированных языков}

\subsubsection{Процесс отображения пользовательского интерфейса}


\subsection{Существующие решения}
