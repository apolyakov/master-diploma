\section{Требования к спецификации и компилятору языка разработки мобильных приложений}
\label{requirements-section}
В данном разделе представлены требования к современному средству
разработки мобильных приложений, собранные на основе обзора существующих
решений, а также сводная таблица соответствия существующих решений собранным
требованиям. Предъявляемые требования были разделены на две группы:
функциональные и нефункциональные.


\subsection{Функциональные требования}
\subsubsection*{Оптимизация отрисовки графического интерфейса}
Отрисовка графического интерфейса мобильного приложения является
ресурсоёмкой задачей. Так, для современного мобильного устройства,
обладающего следующими характеристиками:
\begin{itemize}
	\item разрешение экрана $1644 \times 3840$ точек;
	\item обновление кадра с частотой $120$ Гц;
\end{itemize}
пропускная способность графического конвейера, при условии хранения цвета
одного пикселя в четырёх байтах, должна составлять\\$1644 * 3840 * 4 * 120 = 3030220800$
байт в секунду, что составляет около $2.8$ гигабайт. При условии
необходимости обновления кадров с частотой $120$ Гц, каждый следующий кадр
должен быть подготовлен к отрисовке не позднее, чем через $8$ мс после
отрисовки предыдущего. Подобная нагрузка на мобильное устройство может
негативно сказаться на пользовательском опыте: уменьшение времени работы
устройства от одного заряда аккумулятора, потеря плавности смены кадров
ввиду недостаточной производительности устройства.

По сравнению с алгоритмом, представленным в пункте
~\ref{section:render-pipeline}, использование типовой информации о
графических компонентах позволяет сократить количество операций, необходимых
для обновления кадра. Такая возможность, однако, подразумевает
тесную интеграцию компилятора языка разработки приложений, имеющего
эту информацию, в процесс отрисовки интерфейса.

\subsubsection*{Реактивные обновления пользовательского интерфейса}
Реактивное программирование --- парадигма программирования, ориентированная
на потоки данных и распространение изменений. Это означает cуществование
возможности выражения статических и динамических потоков данных, а также то,
что нижележащая модель исполнения должна автоматически распространять
изменения в рамках определённого потока данных. В контексте
разработки графического интерфейса, под реактивностью понимается
автоматическое обновление пользовательского интерфейса при изменении
данных, помеченных каким-либо образом.

\subsection{Нефункциональные требования}
\subsubsection*{Декларативность описания пользовательского интерфейса мобильного приложения}
В отличие от императивного программирования интерфейсов, в котором
разработчик задаёт как именно необходимо построить интерфейс, декларативное
описание интерфейсов подразумевает от разработчика указание того, что он
хочет увидеть на экране вне зависимости то того, каким образом интерфейс
будет построен.

\subsubsection*{Предоставление отладочных возможностей}
Отладка программ --- неизбежный процесс, возникающий при разработке
большого и сложного программного обеспечения. Существуют несколько видов
отладки: отладочный вывод, трассировка, использование отладчика.
Наиболее совершенным способом отладки является применение
специальных инструментов --- отладчиков. Однако, несмотря на это,
существуют ситуации, когда использование отладчика невозможно по каким-либо
причинам. Исходя из этого, современный язык разработки мобильных приложений
должен предоставлять пользователям различные виды отладки для решения
разного спектра проблем.

\subsubsection*{Кроссплатформенная разработка}
Возможность разработки и поддержки единой кодовой базы мобильного приложения
уже многие годы является потребностью разработчиков. Такая возможность
сказывается на стоимости разработки приложения, что, в свою очередь,
через стоимость конечного продукта влияет и на пользовательский опыт,
получаемый потребителем от взаимодействия с приложением.

Под кроссплатформенностью понимается возможность использования единой
кодовой базы приложения для его компиляции под различные целевые платформы,
например, \name{Android}, \name{iOS}, без необходимости какого-либо
изменения бизнес-логики или структуры графического интерфейса приложения.

\subsubsection*{Поддержка интегрированной средой разработки}
Интегрированная среда разработки стала неотъемлемой частью процесса
создания комплексных программных систем. Мобильные приложения являются
примерами таких систем, поскольку зачастую при их разработке требуется
не только окружение для работы с исходным кодом, но и для работы с
ресурсами приложения, базой данных и так далее.

\subsection{Соответствие существующих решений собранным требованиям}
В таблице~\ref{existing-solutions-table} отображено соответствие
популярных существующих решений собранным требованиям к языку разработки
мобильных приложений и его компилятору. Как видно из таблицы, ни одно
из существующих решений не соответствует всем собранным в данной работе
требованиям.
\begin{table}[h]
	\begin{tabular}{|c|c|c|c|c|}
		\hline
		
		& \thead{Dart /\\Flutter} & \thead{Kotlin DSL /\\Jetpack\\Compose} &
		 \thead{JavaScript /\\React Native} & \thead{Swift /\\SwiftUI} \\
		
		\hline
		\makecell{Оптимизация отрисовки\\графического интерфейса}
		& -- & -- & -- & + \\
		
		\hline
		\makecell{Реактивные обновления\\пользовательского интерфейса}
		& + & + & + & + \\
		
		\hline
		\makecell{Декларативность описания\\пользовательского интерфейса}
		& + & + & + & + \\
		
		\hline
		\makecell{Кроссплатформенная разработка}
		& + & + & + & -- \\
		
		\hline
		\makecell{Предоставление отладочных\\возможностей}
		& + & + & + & + \\
		
		\hline
		\makecell{Поддержка интегрированной\\средой разработки}
		& + & + & + & + \\
		
		\hline
        \end{tabular}
        \caption{Таблица соответствия существующих решений собранным требованиям}
        \label{existing-solutions-table}
\end{table}
