\section{Требования к языку разработки мобильных приложений}
\label{requirements-section}
В этом разделе представлены требования к языку разработки мобильных приложений
наряду с кратким разъяснением причин, по которым они были выдвинуты.
Предъявляемые требования были разделены на три группы: функциональные,
нефункциональные и бизнес-требования.


\subsection{Функциональные требования}
\subsubsection*{Бесшовные реактивные обновления пользовательского интерфейса}

\subsubsection*{"Горячая замена" (\textit{hot-reload})}

\subsubsection*{Кроссплатформенная разработка}

\subsubsection*{Предоставление отладочных возможностей}


\subsection{Нефункциональные требования}
\subsubsection*{Декларативность описания пользовательского интерфейса мобильного приложения}

\subsubsection*{Поддержка интегрированной средой разработки}


\subsection{Бизнес-требования}
\subsubsection*{Поддержка устройств с высоким показателем плотности пикселей на дюйм}
%\textit{Пиксель} --- наименьший логический двумерный элемент цифрового
%изображения в растровой графике, а также наименьшая единица растрового
%изображения, получаемого с помощью графических систем вывода информации,
%таких как компьютерные мониторы и экраны смартфонов.
%\textit{Плотность пикселей} --- это количество пикселей, вмещающееся в
%определённом физическом размере, например в дюйме.


%Современные мобильные устройства, такие как смартфоны и планшеты, могут
%обладать высоким показателем плотности пикселей~\cite{device-ppi-stat}.
\subsubsection*{Поддержка устройств с высоким показателем кадровой частоты}
