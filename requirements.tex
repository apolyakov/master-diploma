\section{Требования к спецификации и компилятору языка разработки мобильных приложений}
\label{requirements-section}
В данном разделе представлены требования к современному средству
разработки мобильных приложений, собранные на основе обзора существующих
решений, а также сводная таблица соответствия существующих решений собранным
требованиям. Предъявляемые требования были разделены на две группы:
функциональные и нефункциональные.


\subsection{Функциональные требования}
\subsubsection*{Оптимизация отрисовки графического интерфейса во время компиляции мобильного приложения}
Отрисовка графического интерфейса мобильного приложения является
ресурсоёмкой задачей. Так, для современного мобильного устройства,
обладающего следующими характеристиками:
\begin{itemize}
	\item разрешение экрана $1644 \times 3840$ точек;
	\item обновление кадра с частотой $120$ Гц;
\end{itemize}
пропускная способность графического конвейера, при условии хранения цвета
одного пикселя в четырёх байтах, должна составлять\\$1644 * 3840 * 4 * 120 = 3030220800$
байт в секунду, что составляет около $2.8$ гигабайт. Такая
нагрузка на мобильное устройство недопустима ввиду скромных возможностей
охлаждения таких устройств, а также ввиду ограниченного объема заряда
аккумуляторной батареи.

Более того, при условии необходимости обновления кадров с частотой $120$ Гц,
новый кадр должен быть доставлен движку отрисовки не позднее, чем через
$8$ мс после отрисовки предыдущего. Если флагманские мобильные устройства
и способны обеспечить должную производительность классического подхода к
обновлению интерфейса, представленного в пункте~\ref{section:render-pipeline},
за счёт увеличенного энергопотребления, то менее мощные устройства на это
уже не способны.

Исходя из вышеописанного, становится очевидной важность
различных оптимизаций отрисовки интерфейса. В особенности тех, что
могут быть применены во время компиляции приложения.

\subsubsection*{Реактивные обновления пользовательского интерфейса}
Реактивное программирование --- парадигма программирования, ориентированная
на потоки данных и распространение изменений. Это означает cуществование
возможности выражения статических и динамических потоков данных, а также то,
что нижележащая модель исполнения должна автоматически распространять
изменения благодаря в рамках определённого потока данных. В контексте
разработки графического интерфейса, под реактивностью понимается
автоматическое обновление пользовательского интерфейса при изменении
данных, помеченных как реактивные.

\subsection{Нефункциональные требования}
\subsubsection*{Декларативность описания пользовательского интерфейса мобильного приложения}
В отличие от императивного программирования интерфейсов, в котором
разработчик задаёт как именно необходимо построить интерфейс, декларативное
описание интерфейсов позволяет разработчику указать то, что он хочет увидеть
на экране, не заботясь о том, каким образом интерфейс будет построен.

\subsubsection*{Предоставление отладочных возможностей}
Отладка программ --- неизбежный процесс, возникающий при разработке
большого и сложного программного обеспечения. Существуют несколько видов
отладки: отладочный вывод, трассировка, использование отладчика.
Наиболее совершенным способом отладки является применение
специальных инструментов --- отладчиков. Однако, несмотря на это,
существуют ситуации, когда использование отладчика невозможно по каким-либо
причинам. Поэтому современный язык разработки мобильных приложений должен
предоставлять пользователям различные виды отладки для решения разного
спектра проблем.

\subsubsection*{Кроссплатформенная разработка}
Возможность разработки и поддержки единой кодовой базы мобильного приложения
уже многие годы является потребностью разработчиков. Такая возможность
сказывается на стоимости разработки приложения, что, в свою очередь,
через стоимость конечного продукта влияет и на пользовательский опыт,
получаемый потребителем от взаимодействия с приложением.

\subsubsection*{Поддержка интегрированной средой разработки}
Интегрированная среда разработки стала неотъемлемой частью процесса
разработки комплексных программных систем. Мобильный приложения являются
примерами таких систем, поскольку зачастую при их разработке требуется
не только окружение для работы с исходным кодом, но и для работы с
ресурсами приложения, базой данных и так далее.


\subsection{Соответствие существующих решений собранным требованиям}
В таблице~\ref{existing-solutions-table} отображено соответствие
популярных существующих решений собранным требованиям к языку разработки
мобильных приложений и его компилятору. Как видно из таблицы, ни одно
из существующих решений не соответствует всем собранным в данной работе
требованиям.
\begin{table}[h]
	\begin{tabular}{|c|c|c|c|c|}
		\hline
		
		& \thead{Dart/\\Flutter} & \thead{Kotlin\\UI DSL} &
		 \thead{React\\Native} & \thead{Swift/\\SwiftUI} \\
		
		\hline
		\makecell{Оптимизация отрисовки\\графического интерфейса\\
		во время компиляции\\мобильного приложения}
		& -- & -- & -- & + \\
		
		\hline
		\makecell{Реактивные обновления\\пользовательского интерфейса}
		& + & + & + & + \\
		
		\hline
		\makecell{Декларативность описания\\пользовательского интерфейса}
		& + & + & + & + \\
		
		\hline
		\makecell{Кроссплатформенная разработка}
		& + & + & + & -- \\
		
		\hline
		\makecell{Предоставление отладочных\\возможностей}
		& + & + & + & + \\
		
		\hline
		\makecell{Поддержка интегрированной\\средой разработки}
		& + & + & + & + \\
		
		\hline
        \end{tabular}
        \caption{Сводная таблица соответствия существующих решений собранным требованиям}
        \label{existing-solutions-table}
\end{table}
