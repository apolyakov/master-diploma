\section{Требования к спецификации и компилятору языка разработки мобильных приложений}
\label{requirements-section}
В данном разделе представлены требования к современному средству
разработки мобильных приложений, собранные на основе обзора существующих
решений, а также сводная таблица соответствия существующих решений собранным
требованиям. Предъявляемые требования были разделены на две группы:
функциональные и нефункциональные.


\subsection{Функциональные требования}
\subsubsection*{Оптимизация отрисовки графического интерфейса с использованием статической информации о программе}

\subsubsection*{Реактивные обновления пользовательского интерфейса}


\subsection{Нефункциональные требования}
\subsubsection*{Декларативность описания пользовательского интерфейса мобильного приложения}

\subsubsection*{Предоставление отладочных возможностей}

\subsubsection*{Кроссплатформенная разработка}

\subsubsection*{Поддержка интегрированной средой разработки}


\subsection{Соответствие существующих решений собранным требованиям}
\begin{table}[h]
	\begin{tabular}{|c|c|c|c|c|}
		\hline
		
		& \textit{VueJS} & \textit{SwiftUI} &
		\textit{Flutter} & \textit{Flow9} \\
		
		\hline
		\makecell{Оптимизация отрисовки\\графического интерфейса\\
		с использованием\\статической информации\\о программе}
		& -- & + & -- & -- \\
		
		\hline
		\makecell{Реактивные обновления\\пользовательского интерфейса}
		& + & + & -- & + \\
		
		\hline
		\makecell{Декларативность описания\\пользовательского интерфейса}
		& + & + & + & + \\
		
		\hline
		\makecell{Кроссплатформенная разработка}
		& + & -- & + & + \\
		
		\hline
		\makecell{Предоставление отладочных\\возможностей}
		& + & + & + & + \\
		
		\hline
		\makecell{Поддержка интегрированной\\средой разработки}
		& + & + & + & + \\
		
		\hline
        \end{tabular}
        \caption{Сводная таблица соответствия существующих решений собранным требованиям}
        \label{existing-solutions-table}
    \end{table}

Как видно из таблицы~\ref{existing-solutions-table}, на данный момент
наиболее популярные языки разработки мобильных приложений не удовлетворяют
всем требованиям, собранным в данной работе.
